\documentclass[a4paper,10pt]{article}

\usepackage{graphicx}
\usepackage[ansinew]{inputenc}
\usepackage[spanish]{babel}
\usepackage{listings} 
\usepackage{float}
\usepackage[justification=centering]{caption}


\title{		\textbf{Trabajo pr�ctico \#1: conjunto de instrucciones MIPS}}

\author{	Santiago Fernandez, \textit{Padr�n Nro. 94.489}                     \\
            \texttt{ fernandezsantid@gmail.com }                                              \\[2.5ex]
            Francisco Landino, \textit{Padr�n Nro. 94.475}                     \\
            \texttt{ landinofrancisco@gmail.com }                                              \\[2.5ex]
            Matias Duarte, \textit{Padr�n Nro. 92.186}                     \\
            \texttt{ duarte.mati@gmail.com }                                              \\[2.5ex]
            \normalsize{2do. Cuatrimestre de 2014}                                      \\
            \normalsize{66.20 Organizaci�n de Computadoras  $-$ Pr�ctica Jueves}  \\
            \normalsize{Facultad de Ingenier�a, Universidad de Buenos Aires}            \\
       }
\date{}

\begin{document}

\maketitle
\thispagestyle{empty}   % quita el n�mero en la primer p�gina


\begin{abstract}
Este trabajo pr�ctico trata de una versi�n en lenguaje C de un programa que lee un archivo de texto que contiene tags posiblemente anidados, y 
verifica que la estructura sea correcta. Adem�s, el objetivo fue familiarizarse con el conjunto de instrucciones MIPS32 y el concepto de ABI.
\end{abstract}
\pagebreak 

\tableofcontents
\pagebreak

\section{Introducci�n}

\pagebreak 

\section{Desarrollo}


\subsection{Recursos y Portabilidad}

Uno de los objetivos del trabajo pr�ctico es poder probar la portabilidad del
programa en diferentes entornos. En el enunciado se ped�a que el programa se pudiera ejecutar en NetBSD[4]/pmax (usando el simulador GXemul[5]) y en la versi�n
de Linux (Knoppix, RedHat, Debian, Ubuntu) usada para correr el simulador,
Linux/i386. En particular, se lo prob� en Ubuntu 14.04. En GXemul se corri� una m�quina de arquitectura MIPS cuyo sistema operativo era una versi�n reciente de NetBSD/pmax. La transferencia de archivos entre la m�quina host y la guest se hizo mediante \textit{SSH}. Se procedi� de la siguiente manera:

Para trabajar con el GXemul se procedi� primero creando una nueva interfaz de red (debe crearse cada vez que se inicia el \textit{host} y con permisos de administrador):
\\

\emph{hostOS\$ sudo ifconfig lo:0 172.20.0.1}
\\

Luego se ejecut� el GXemul en modo X:
\\

\emph{hostOS\$ ./xgxemul -e 3max -d netbsd-pmax.img -x}
\\

Una vez ya ingresado con el usuario y la contrase�a en la m�quina simulada,
se cre� un t�nel reverso para saltear las limitaciones propias del GXemul:
\\

\emph{guestOS\$ ssh -R 2222:127.0.0.1:22 usuario@172.20.0.1}
\\

A partir de ese momento y dejando lo anterior en segundo plano, ya se puedo
trabajar mediante SSH de manera m�s c�moda:
\\

\emph{hostOS\$ ssh -p 2222 root@127.0.0.1}
\\

\subsection{Implementaci�n}



\subsection{Compilaci�n}

Para compilar el trabajo pr�ctico, se tiene que ejecutar la siguiente l�nea en la terminal de Linux.\\
\\
\emph{gcc -Wall -pedantic -std=c99 main.c auxiliares.c -I / -o tp0}\\

Para simplificar este proceso, y no tener que escribir el comando entero, se cre� un archivo \emph{compile.sh}, mediante el cual, con solo escribir \emph{./compile.sh}, se compila el trabajo pr�ctico.

\subsection{Corrida de Pruebas}
\\

\begin{lstlisting}

\end{lstlisting}


\subsection{C�digo}
\subsubsection{C�digo en C}

\begin{lstlisting}[language=C, columns=flexible, breaklines=true, frame=single, title=Listing : auxiliares.c]

\end{lstlisting}

\begin{lstlisting}[language=C, columns=flexible, breaklines=true, frame=single, title=Listing : main.c]

\end{lstlisting}

\begin{lstlisting}[language=C, columns=flexible, breaklines=true, frame=single, title=Listing : auxiliares.h]

\end{lstlisting}

\newpage
\section{Conclusiones}

\\

\newpage
\section{Referencias}
\noindent[1] GXemul, http://gavare.se/gxemul/.\\ \relax
[2] The NetBSD project, http://www.netbsd.org/.\\ \relax
[3] System V application binary interface, MIPS RISC processor supplement(third edition). Santa Cruz Operations, Inc.\\ \relax
[4] func call conv.pdf, en el area de Material de los archivos del grupo de Yahoo.\\ \relax

\end{document}